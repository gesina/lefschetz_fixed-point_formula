\documentclass[english,headsepline=0.25pt]{scrartcl}
\usepackage{fontspec}
\usepackage{babel}
\usepackage{csquotes}
\usepackage[backend=biber]{biblatex}
\bibliography{lefschetz_formula.bib}
\usepackage{scrlayer-scrpage}
\usepackage{mathtools, amssymb, amsthm, dsfont}
\usepackage{enumitem}
\setlist[description]{font=\normalfont\bfseries}
\setlist[enumerate]{label=(\arabic*)}
\usepackage{tikz-cd}

\usepackage{hyperref}
\hypersetup{
  pdfauthor=Gesina Schwalbe,
  pdftitle=Lefschetz fixed-point formula,
  colorlinks=true, linkcolor=blue, citecolor=cyan, urlcolor=blue
}

% THEOREMS
\theoremstyle{definition}
\newtheorem{Def}{Definition}[section]
\newtheorem{DefLem}[Def]{Definition/Lemma}
\newtheorem{Prop}[Def]{Proposition}
\newtheorem{Thm}[Def]{Theorem}
\newtheorem*{Thm*}{Theorem}
\newtheorem{Lem}[Def]{Lemma}
\newtheorem{Cor}[Def]{Corollary}
\theoremstyle{remark}
\newtheorem{Rem}[Def]{Remark}
\newtheorem{Rev}[Def]{Reminder}
\newtheorem{Ex}[Def]{Example}

% SHORTENINGS
\newcommand*{\N}{\mathds{N}}
\newcommand*{\Z}{\mathds{Z}}
\newcommand*{\Q}{\mathds{Q}}
\newcommand*{\R}{\mathds{R}}
\newcommand*{\C}{\mathds{C}}
\newcommand*{\fF}{\mathds{F}} % finite field
\newcommand*{\Zmod}[1]{\Z/#1} % Z/nZ
\newcommand*{\Zl}{\Z_l} % l-adic completion of Z
\newcommand*{\Ql}{\Q_l} % quotient field of \Zl
\newcommand*{\F}{\mathcal{F}} % sheaf F
\newcommand*{\G}{\mathcal{G}} % sheaf G
\DeclareMathOperator{\Char}{char} % characteristic of a field
\DeclareMathOperator{\codim}{codim} % codimension
\DeclareMathOperator*{\dirlim}{dirlim} % direct limit
\newcommand*{\End}[1]{\text{End}(#1)} % Endomorphisms
\DeclareMathOperator{\Ext}{Ext} % Ext functor
\DeclareMathOperator{\Tor}{Tor} % Tor functor
\DeclareMathOperator{\Hom}{Hom} % Hom functor
\newcommand*{\id}{\text{id}} % identity
\DeclareMathOperator{\Sh}{Sh} % sheaves over a space
\newcommand*{\idest}{i.e.\ }
\newcommand*{\forexample}{e.g.\ }
\newcommand*{\argreplacement}{\,\cdot\,} % argument replacement for functors/functions

\newcommand*{\M}{\Lambda}
% \M_l(r)(R) = {
% for r=0 \ZlZ;
% for r>0 \mu_l(R)^{\otimes r};
% for r<0 Hom_{\Zmod{l}}(\mu_l(R)^{\otimes -r}, \Zmod{l}) }
\newcommand*{\Tr}{\text{tr}} % trace isomorphism
\newcommand*{\tr}[1]{\Tr\left(#1\right)} % trace isomorphism
\newcommand*{\intProd}[2]{{#1\cdot#2}} % intersec. product in the Chow ring
\newcommand*{\intNum}[1]{{\left\langle{#1}\right\rangle}} % intersection number (= tr(intersection product))
\newcommand*{\Graph}[1]{{\Gamma_{#1}}} % graph of a map
\newcommand*{\Diag}[1]{{\Delta_{#1}}} % diagonal (both map and subscheme)
\newcommand*{\trace}[2]{{\text{trace}\left(#1 \,\middle|\, #2 \right)}} % trace of map (#1) in vector space (#2)
\DeclareMathOperator{\CH}{CH} % Chow group functor
\DeclareMathOperator{\Div}{Div} % Divisor group
\DeclareMathOperator{\Pic}{Pic} % Picard group
\DeclareMathOperator{\CL}{cl} % cycle class map
\newcommand*{\cl}[2]{{\CL_{#1}\left(#2\right)}} % cycle class map
\newcommand*{\one}[1]{{1_{#1}}}%{H^*(#1)}}
\let\altphi\phi
\renewcommand*{\phi}{\varphi}
\newcommand*{\Poincare}{{\text{Poincaré \ref{poincare}}}}
\newcommand*{\Hc}{{H_c}} % cohomology with compact support
\newcommand*{\dual}[1]{\check{#1}} % dual basis vector

\begin{document}
% HEAD/FOOT
\clearpairofpagestyles
\ohead{Universität Regensburg}
\ihead{Gesina Schwalbe}
\cfoot*{\pagemark}

% TITLE
\title{Poincaré Duality and\\
  Lefschetz Fixed-Foint Formula}
\subject{Seminar:
  Deligne's proof of the Weil Conjecture%\\by Federico Binda, Prof. Dr. Moritz Kerz
}
% \subtitle{Lecture Notes}
\author{Gesina Schwalbe\\
  \normalsize
  Supervisors:\\\normalsize
  \href{https://fedebinda.wordpress.com}{Dr.\,Federico Binda}, 
  \href{http://www.mathematik.uni-regensburg.de/kerz/}{Prof.\,Dr.\,Moritz Kerz}
}
\date{\normalsize\today\\
  University of Regensburg\\
}
\maketitle
\tableofcontents

\section{Motivation}
From topology the following theorem is known that relates the fixed
points of a morphism (counted with multiplicity) with homology:
\begin{Thm*}[Lefschetz-Hopf]
  Let $X$ be a compact, smooth manifold and
  $\phi\colon X\to X$ a morphism. Then
  \begin{gather*}
    \sum_{x\in\operatorname{Fix}(\phi)} \operatorname{i}(\phi,x)
    =\sum_{r\in\N}(-1)^r \trace{\phi^*}{H_\text{sing}^r(X,\Q)}\;,
  \end{gather*}
  where $\operatorname{Fix}(\phi)$ are the fixed points of $\phi$ and
  $\mathrm i(\phi,x)$ is the multiplicity of a fixed point $x$ of $\phi$
  in $X$.
\end{Thm*}
The proof uses the Poincaré duality theorem which says
\begin{Thm*}[Poincaré]
  For a smooth manifold $X$ and $r\in\Z$ holds
  \begin{gather*}
    H_{c\text{,sing}}^r(X,\Z) \cong H_{\dim(X)-r}^{\text{sing}}(X,\Z)
  \end{gather*}
  where $H_{c\text{,sing}}^r(X,\Z)\coloneqq
  \dirlim_{Z\subset X \text{cpt.}}H_{Z,\text{sing}}^r(X,\Z)$
  is cohomology with compact support for $X$,
  with $H_{Z,\text{sing}}^r(X,\Z)$ being the $r$th relative
  $\Z$-cohomology of the pair $(X,X\setminus Z)$.
\end{Thm*}
The aim of this lecture is to introduce and proof an analogue
statement of the Lefschetz formula for schemes instead of topological
spaces, as well as to give an overview of the necessary preliminaries.
Some convenient correspondences are
\begin{itemize}
\item an algebraically closed field $k$ instead of $\R$ as ground field,
\item $k$-\emph{variety}
  (\idest separated, geometrically reduced scheme of finite type over $k$)
  instead of \emph{$\R$-manifold},
\item \emph{complete} instead of \emph{compact},
\item \emph{non-singular} instead of \emph{smooth},
\item \emph{morphism of varieties} instead of \emph{morphism of manifolds},
\item \emph{étale cohomology} instead of \emph{singular homology}
  (mind that they coincide for non-singular varieties over $\C$ and
  finite coefficient group \cite[see][Thm.~I.21.1]{milne}),
\item \emph{constant sheaf with value $G$}
  as coefficients instead of an abelian group $G$
  (written $G\in\Sh(X)$ where
  $\Sh(X)$ are the sheaves of abelian groups over $X$)
\item \emph{intersection number} $\intNum{\intProd{\Graph{\phi}}{\Diag{X}}}$
  (also counting fixed points of $\phi$ with multiplicity)
  instead of $\sum_{x\in\operatorname{Fix}(\phi)}\mathrm{i}(\phi,\cdot)$.
\end{itemize}
We also need to consider modified coefficients since étale cohomology
does not behave well for constant coefficient sheaves of non-finite
abelian groups. However, passing to limits we can consider for $X$
(étale) scheme the étale cohomology with coefficients in $\Ql$ as follows
\begin{gather*}
  H^r(X,\Ql)\coloneqq
  \left(\lim_{n\to\infty}H^r(X,\Zmod{l^n})\right)\otimes_{\Zl}\Ql
\end{gather*}
Thus the final statement of the Lefschetz fixed point formula shall be
(see \ref{lefschetzthm}):
\begin{Thm}[Lefschetz]\label{lefschetzthm:motivation}
  Let $k$ be an algebraically closed field, $X$ a non-singular,
  complete $k$-variety, and let $\phi\colon X\to X$ be an
  endomorphism. Then
  \begin{gather*}
    \intNum{\intProd{\Graph{\phi}}{\Diag{X}}}
    = \sum_{r\in\N}(-1)^r\trace{\phi^*}{H^r(X,\Ql)}
  \end{gather*}
\end{Thm}

We instantly get the following two applications:
\begin{description}
\item[Zeta function] For
  $k=\fF_{q}$ a finite field ($q\in\N$),
  $X$ a non-singular variety over $k$ and
  $\phi=F^n$ the $n$th Frobenius homomorphism
  we get
  \begin{gather*}
    \#X(\fF_{q^n}) = \intNum{\intProd{\Graph\phi}{\Diag X}}
  \end{gather*}
  which is essential to further investigate the Zeta function of $X$.
\item[Euler characteristic] Taking $k$ and $X$ as in the Theorem and
  $\phi=\Delta$ we see that the étale Euler characteristic of $X$ is
  just its self-intersection number, \idest
  \begin{gather*}
    \intNum{\intProd{\Delta}{\Delta}} = \chi_{\text{ét}}(X)
  \end{gather*}
\end{description}

For the proof of \ref{lefschetzthm:motivation}
some further nice properties of étale cohomology for
(nice) varieties will be needed, such as
a Künneth isomorphism (see \ref{kuennethiso}),
a cycle class map that gives a connection to the Chow ring and
intersection product (see \ref{def:cycleclassmap}),
Poincaré duality with a trace map (see \ref{poincare}),
as well as pushforwards on cohomology (see \ref{def:pushforward}).
To establish these, one needs for étale cohomology the notion of
cohomology with compact support (see \ref{def:cptcohomology}),
relative cohomology (see \ref{def:relcoh}),
long exact sequences of (compact) pairs (see \ref{lesrelcoh}),
Gysin sequences (see \ref{gysiniso}),
and most of all a cup product, \idest a graded ring structure
(see \ref{def:cupproduct}).
All of this will be introduced in the following.

\section{Preliminaries on étale Cohomology}
It will be assumed that the reader knows about the definition and basic
properties of étale cohomology, otherwise consult \forexample
\cite{jannsenetale}, \cite{milne}, \cite{milnebook}.

If not stated otherwise, for a (Zariski) scheme $X$ and a (Zariski)
sheaf of modules $\F$ over $X$ we denote with $H^r(X,\F)$ the $r$th
étale cohomology group of the étale sheaf $\F$ over the étale scheme
of $X$
\cite[compare][Chap.~I.6, Examples of Sheaves on $X_\text{ét}$]{milne}.
Also recall that for $\F$ torsion sheaf and $X$ complete these groups
are finite \cite[][Thm.~I.19.1]{milne}.

\subsection{Étale Cohomology with compact support}
Throughout this section let $k$ be an algebraically closed field,
$U$ a variety over $k$,
and $\F$ be a torsion sheaf over $U$.

The definition of cohomology with compact support as
$\left(\dirlim_{Z\subset X \text{cpt.}}H_Z^r(X,\Z)\right)$
cannot be copied from topology but has to be adjusted.
However, we will see later (namely \ref{def:etapoint}),
that---at least in degree 0---the following definition
and the expression above coincide.

First recall the following
\begin{Rev}[Nagata compactification]\label{nagata}
  There always exists a complete $k$-variety $X$ and an open, dense
  immersion $j\colon U\to X$. Such a $j$ is also called
  \emph{compactification} of $U$.
\end{Rev}
\begin{Rev}[Extension by zero]\label{def:extbyzero}
  For an open immersion $j\colon U\to X$ into
  a scheme $X$, $j_!$ (called the \emph{extension by zero} of $j$)
  is defined as
  \begin{gather*}
    j_!\colon \Sh(U)\to \Sh(X)\;,\qquad
    j_!\F \coloneqq \text{sheafification of}\quad
    V\mapsto\begin{cases}
      \F(V) & V\subset U\\
      0     & \text{else}
    \end{cases}
  \end{gather*}
\end{Rev}

\begin{DefLem}\label{def:cptcohomology}
  Let $j\colon U\to X$ be a compactification of $U$. Then the
  cohomology with compact support in degree $r\in\Z$ is defined as
  \begin{gather*}
    \Hc^r(U,\F) \coloneqq H^r(X,j_!(\F))
  \end{gather*}
  This definition is independent of the choice of compactification.
  \begin{proof}[proof (scetch)]
    The set of compactifications of $U$ is partially ordered by proper
    morphisms:
    For two compactifications $j_1\colon U\to X_1$,
    $j_2\colon U\to X_2$ the closure
    $X\coloneqq\overline{(j_1,j_2)(U)}$ in $X_1\times_X X_2$ is again
    a compactification and the projections on $X_1$, $X_2$ are proper.
    Thus one only needs to check $\Hc^r(X,j_!\F)\cong\Hc^r(X',j_!'\F)$
    for compactifications $j$, $j'$ and a proper map $\pi$ such that
    the following diagramm commutes
    \begin{center}
      \begin{tikzcd}
        U\arrow[r,"j"]\arrow[dr,"j'"{left}, bend right=20]
        & X\arrow[d,"\pi"]\\
        & X'
      \end{tikzcd}
    \end{center}
    This can be shown using a version of the Leray spectral
    sequence and the proper base change theorem
    \cite[][Thm. I.18.2]{milne}.
  \end{proof}
\end{DefLem}

\begin{Rem}
  Note that this is no derived functor!
  However, since $j_!$ is exact for a compactification $j$ of $U$, we
  still get a long exact sequence of cohomology for any short exact
  sequence of sheaves on $U$
  (just use the definition and the long exact sequence of cohomology for
  $H^*(X,\cdot)$).
\end{Rem}

\subsection{Cup Product and Künneth formula}
Again let $k$ be an algebraically closed field throughout the
section.

The foundation of all further investigations will be the graded
ring structure on the sum of the cohomology groups in all degrees
(with appropriate coefficients). The product will be a cup product
denoted $\cup$ to which we here give an approach via pairings of
Ext-groups. 
However, there are concrete descriptions using Čech-cohomology
\cite[see][Chap.~I.10 and Chap.~I.22, Cup-products]{milne}
or using the Künneth isomorphism defined later on (see
\ref{kuennethiso}, \ref{rem:cupwithkuennethiso}).

\begin{Rem}[Yoneda product]\label{yonedaproduct}
  For an abelian category $\mathcal{A}$ with enough injectives and
  any objects $A,B,C$ of $\mathcal{A}$, $r,s\in\N$ there exists a pairing
  \begin{align*}
    \Ext^r(A,B)\otimes\Ext^s(B,C) &\overset{\cup}{\longrightarrow}
                                    \Ext^{r+s}(A,C)
                                    % \intertext{which is induced by}
    \\
    \text{induced by}\qquad
    \Hom(A,B)\otimes\Hom(B,C) &\longrightarrow \Hom(A,C)\;,\quad
                                f\otimes g\mapsto f\circ g
  \end{align*}
  \begin{proof}[idea for construction]
    The elements of an $r$th Ext-group can be described as extensions
    (\idest sequences) of length $r$.
    Define $\cup$ to be the concatenation of two such sequences. 
  \end{proof}
\end{Rem}

\begin{Prop}[Cup product]\label{def:cupproduct}
  For a $k$-variety $X$, and $\F,\G\in\Sh(X)$ get for all $r,s\in\Z$
  natural pairings
  \begin{gather*}
    H^r(X,\G)\otimes H^r(X,\F) \overset{\cup}\longrightarrow
    H^{r+s}(X,\F\otimes\G)
  \end{gather*}
  This is associative as well as graded commutative, \idest $x\cup y =
  (-1)^{rs}y\cup x$ for $x\in H^r(X,\F)$, $y\in H^s(X,\G$
  \cite[][Rem.~V.1.18]{milnebook}.
  \begin{proof}[idea for construction]
    The cup product is the map making the following diagram commute
    where the lower row is the \nameref{yonedaproduct}
    \begin{center}
      \begin{tikzcd}[column sep=tiny]
        H^r(X,\F) \arrow[r, phantom, "\otimes"{description}] \arrow[d, equals]&
        H^r(X,\G) \arrow[r] \arrow[d]
        & H^{r+s}(X,\F\otimes\G) \arrow[d,equals]\\
        \Ext_X^r(\mathcal O_X,\F) \arrow[r, phantom, "\otimes"{description}]&
        \Ext_X^s(\F,\F\otimes\G) \arrow[r]
        & \Ext_X^{r+s}(\mathcal O_X,\F\otimes\G)
      \end{tikzcd}
    \end{center}
    This already implies associativity (compare the construction of the
    Yoneda product in \ref{yonedaproduct}).
    
    There is also a more general approach to define cup products for
    derived functors. This states, that for any right exact functor
    $\Gamma\colon \Sh(X)\to\text{AbGrp}$ fulfilling certain nice
    properties, a bifunctor
    $\Gamma(\F)\otimes\Gamma(\G)\to\Gamma(\F\otimes\G)$
    extends to a cup product pairing
    \begin{gather*}
      \operatorname{R}^r\Gamma(\F)\otimes
      \operatorname{R}^s\Gamma(\G) \to
      \operatorname{R}^{r+s}\Gamma(\F\otimes\G)
    \end{gather*}
    \cite[see][Prop. V.1.16]{milnebook}
  \end{proof}
\end{Prop}

We can now proceed with the definition of the graded ring of
cohomology groups by choosing appropriate coefficients.
\begin{Def}\label{def:coefficients}
  Let $X$ be a $k$-variety,
  $n\in\N$ such that $\Char(k)\nmid n$,
  and $r\in\Z$.
  \begin{enumerate}
  \item We denote with $\Lambda\coloneqq\Lambda_n$ the constant sheaf
    with value $\Zmod{n}$ over $X$.
  \item\label{def:moduletwists}
    For a $\Zmod{n}$-module $F$ define the $r$th tensor product
    (or $r$th twist) as
    \begin{align*}
      \F(r)\coloneqq F^{\otimes r}
      &\coloneqq \begin{cases}
        F^{\otimes r} &r>0\\
        \Zmod{n} &r=0\\
        \Hom_{\Zmod{n}}(F^{\otimes (-r)},\Zmod{n}) &r<0
      \end{cases}
    \end{align*}
  \item\label{def:rootsofunity} For an integral domain $R$ let
    \begin{gather*}
      \mu_n(R)\coloneqq\{ u\in R^\times\;|\;u^n=1_R\}
    \end{gather*}
    be the $n$th roots of unity in $R$. 
    Note that if $R$ contains all $n$th roots of unity---\forexample
    contains an algebraically closed field---then the choice of a
    primitive $n$th root of unity determines (non-canonical) isomorphisms
    $\mu_n(R)^{\otimes s}\cong\Zmod{n}$ for all $s\in\Z$.
  \item By choice of $n$, the following sheaf of roots of unity is a
    well-defined, locally free sheaf of $\Lambda$-modules of rank~1
    over $X$
    \begin{gather*}
      \overline\mu_n(X) \coloneqq \text{ sheafification of }\quad
      U\mapsto \mu_n(\Gamma(U,\mathcal O_U))
    \end{gather*}
  \item\label{def:tatetwist}
    We can now (similarly to \ref{def:moduletwists} above) define the
    \emph{$r$th Tate twist} of the sheaf of modules $\Lambda$ as
    \begin{align*}
      \Lambda(r) &\coloneqq \begin{cases}
        \overline\mu_n(X)^{\otimes r} &r>0\\
        \Lambda &r=0\\
        \Hom_{\Lambda}(\overline\mu_n(X)^{\otimes (-r)},\Lambda) &r<0
      \end{cases}
    \end{align*}
    Mind, that we are considering homomorphisms of sheaves.
    Obviously, we have
    $\Lambda(r)^\vee\coloneqq\Hom_\Lambda(\Lambda(r),\Lambda)\cong\Lambda(-r)$.
    Furthermore, for $k$ algebraically closed $\Lambda(r)$ is
    (non-canonically) isomorphic to $\Lambda$ by the explanation in
    \ref{def:rootsofunity} above. 
    However, choosing a primitive $n$th root of unity determines
    isomorphisms $\Lambda\cong\Lambda(s)$ for all $s\in\Z$.
  \item For any $n$-torsion sheaf over $X$ (\idest a $\Lambda$-module)
    we define
    \begin{flalign*}
      \SwapAboveDisplaySkip
      &\text{the $r$th Tate twist of $\F$ as $\Lambda$-module as}
      &\F(r) &\coloneqq \F\otimes_\Lambda \Lambda(r)\;,
      \\
      &\text{the dual sheaf of $\F$ as $\Lambda$-module as}
      &\F(r)^\vee &\coloneqq \Hom_\Lambda(\F,\Lambda(r))\cong
      \F(-r)\;.
      \qquad
    \end{flalign*}
  \end{enumerate}
\end{Def}
One may notice an analogue of Serre twists in sheaf cohomology,
where $\mathcal O_X$-~instead of $\Lambda$-modules are considered.

With the coefficients defined above we can now state the ring
structure as follows:
\begin{DefLem}
  For a $k$-variety $X$, and $n\in\Z$ such that $\Char(k)\nmid n$,
  define for $r\in\Z$
  \begin{align*}
    \SwapAboveDisplaySkip
    H^r(X) \coloneqq H^r(X,\Lambda(r))\;,\qquad
    H^*(X) \coloneqq \bigoplus_{r\in\N}H^r(X)
  \end{align*}
  This is equipped with the following properties:
  \begin{enumerate}
  \item $H^*(X)$ together with the cup product
    $H^*(X)\otimes H^*(X)\xrightarrow{\cup} H^*(X)$
    as in \ref{def:cupproduct} is a graded commutative ring
    (follows directly from the properties of the cup product).
  \item The canonical, natural isomorphism $\Zmod{n}\cong H^0(X)$
    makes $H^*(X)$ a $\Zmod{n}$-Algebra with unit $1_X$,
    which is the image of $1_{\Zmod{n}}$ under this isomorphism.
  \item For any morphism of $k$-varieties $\phi\colon X\to Y$ the
    induced map $\phi^*\colon H^*(Y)\to H^*(X)$ is a homomorphism of
    graded rings
    (follows from the naturality of the cup product).
  \end{enumerate}
\end{DefLem}

\begin{Rem}\label{cohomologyoftwists}
  Consider $X,n,r$ as in \ref{def:coefficients} above, and $k$ an
  algebraically closed field. Then for any $\Lambda$-module $\F$
  (\idest $n$-torsion sheaf over $X$) there is by the comment in
  \ref{def:coefficients}\ref{def:tatetwist} a non-canonical
  isomorphism $\F\cong\F(r)$. However, on cohomology this becomes a
  canonical isomorphism
  \cite[see][p.\,66 and p.\,163]{milnebook}
  \begin{gather*}
    H^r(X,\F)\otimes \Zmod{n}(r) \cong H^r(X,\F(r))
  \end{gather*}
\end{Rem}


A first, heavily simplifying observation is the existence of a Künneth
formula which correlates the cohomology ring of a product with the
cohomology rings of each factor.

\begin{Thm}[Künneth formula]\label{kuennethiso}
  Let $X$ and $Y$ be $k$-varieties
  ($k$ here only needs to be separably closed), and
  $n\in\Z$ with $\Char(k)\nmid n$.
  Further let $\F$ and $\G$ be constructible, flat $\Lambda_n$-modules
  over $X$ respectively $Y$.
  Consider the projection maps
  $p\colon X\times_k Y\to X$ and
  $q\colon X\times_k Y\to Y$ and denote
  $\F\boxtimes\G\coloneqq p^*\F\otimes q^*\G$. 
  \begin{enumerate}
  \item   Then for any $m\in\N$ there is an exact sequence
    \begin{center}
      \begin{tikzcd}[row sep=tiny, column sep=small]
        0 \rar
        &\displaystyle\bigoplus_{\mathclap{r+s=m}}H^r(X,\F)\otimes H^s(Y,\G) \arrow[r, "k"]
        &H^m(X\times_k Y,\F\boxtimes\G) \dar\\
        &&\displaystyle\bigoplus_{\mathclap{r+s=m+1}}\Tor_1^{\Zmod{n}}\left(
          H^r(X,\F), H^s(Y,\G)
        \right) \rar
        &0
      \end{tikzcd}
    \end{center}
    The Künneth-map $k$ is defined by
    \begin{center}
      \begin{tikzcd}[row sep=small]
        \displaystyle\bigoplus_{\mathclap{r+s=m}}
        H^r(X,\F)\otimes H^s(Y,\G)
        \dar&
        \displaystyle\sum_{\mathclap{i}}a_i\otimes b_i
        \arrow[dd,mapsto]\\
        \displaystyle\bigoplus_{\mathclap{r+s=m}}
        H^r(X\times_k Y,q^*\F)\otimes H^s(X\times_k Y,q^*\G)
        \arrow[d,"\cup"]\\
        H^m(X\times_k Y,\F\boxtimes\G)
        &\displaystyle\sum_{\mathclap{i}}p^*a_i \cup q^*b_i
      \end{tikzcd}
    \end{center}
    This statement also holds if one exchanges étale cohomology with
    étale cohomology with compact support.
  \item For $X$ and $Y$ complete the map $k$ is an isomorphism. 
  \end{enumerate}
  \begin{proof}\cite[][Thm.~V.8.5 and Cor.~V.8.13]{milnebook}\end{proof}
\end{Thm}

\begin{Rem}
  For $X$ and $Y$ complete $k$-varieties we obtain a natural
  isomorphism of graded rings, the so-called Künneth isomorphism:
  \begin{align*}
    H^*(X)\otimes H^*(Y) &\rightarrow H^*(X\times_k Y)\\
    a\otimes b &\mapsto p^*(a) \cup q^*(b)
  \end{align*}
  Mind that functoriality makes it compatible with the limit maps
  $\Lambda_{l^{n+1}}(r)\to\Lambda_{l^n}(r)$.
\end{Rem}

\begin{Rem}\label{rem:cupwithkuennethiso}
  The existence of the Künneth isomorphism can also be used for
  a definition of the cup product by identifying $H^*(X\times_k Y)$
  with $H^*(X)\otimes H^*(Y)$. For $x\in H^r(X)$, $y\in H^s(Y)$
  we then define
  \begin{gather*}
    \SwapAboveDisplaySkip
    x\cup y \coloneqq (\Diag{X})^*(x\otimes y)
  \end{gather*}
\end{Rem}

\subsection{The Chow Ring and the Cycle Class Map}
In the statement of the Lefschetz formula the usual intersection
product occurs. The first aim of this section will be to place the
intersection product in a more general setting, more precisely
identify it with the product in the so-called Chow ring. Afterwards we
state the connection between the cohomology and the Chow ring in form
of a natural, graded ring homomorphism.

$k$ is still considered to be an algebraically closed field,
$n\in\Z$ an integer such that $\Char(k)\nmid n$,
and $\Lambda=\Lambda_n$ as in \ref{def:coefficients} above.

\begin{DefLem}
  Let $X$ be a non-singular variety.
  \begin{enumerate}
  \item A \emph{prime cycle} $Z$ of $X$ is an irreducible, closed
    subvariety.
  \item Denote the set of all prime cycles in $X$ of codimension
    $r\in\N$ by $C_p^r(X)$ and define
    \begin{gather*}
      C^r(X) \coloneqq \bigoplus_{C_p^r(X)}\Z
    \end{gather*}
    to be the free abelian group generated by prime cycles of
    codimension $r$.
    Note: $C^1(X)=\Div(X)$.
  \item The intersection
    $Y\cap Y'\coloneqq (\Diag{X})^*(Y\times_X Y')$ of two cycles 
    $Y\in C^r(X)$, $Y'\in C^s(X)$ is said to be
    \begin{description}[font=\normalfont\itshape]
    \item[proper,] if for every irreducible component $Z$ of the
      intersection $Y\cap Y'$ holds
      $\codim(Y)+\codim(Y')=\codim(Z)$,
    \item[transversal in a point,] if the tangent spaces of
      $Y$ and $Y'$ span the tangent space of $Z$ in this point,
    \item[transversal,] if it is so in every point.
    \end{description}
    If $Y$ and $Y'$ intersect properly we define
    the intersection product of $Y$ and $Y'$ as
    \begin{gather*}
      \SwapAboveDisplaySkip
      \intProd{Y}{Y'} \coloneqq
      \sum_{\mathclap{
          \substack{Z\subset Y\cap Y'\\
            \text{irred. comp.}
          }}} 1_\Z\cdot Z
    \end{gather*}
  \item The $r$th \emph{Chow group} of $X$ is defined as
    \begin{gather*}
      \CH^r(X) \coloneqq C^r(X)/\sim
    \end{gather*}
    where $\sim$ denotes rational equivalence
    \cite[see][p.\,426]{hartshorne}.
  \item This equivalence fulfills \emph{Chow's moving lemma}
    \cite[][p.\,427]{hartshorne}:
    For any $Y\in C^r(X)$, $Y'\in C^s(X)$ there are
    $\widetilde Y\in C^r(X)$, $\widetilde Y'\in C^s(X)$ intersecting
    properly, such that $Y\sim\widetilde Y$ and $Y'\sim\widetilde Y'$.
    Thus the intersection product is well defined on equivalence
    classes of the Chow groups and we obtain the \emph{Chow ring}
    \begin{gather*}
      \CH^*(X) \coloneqq \bigoplus_{r\in\N} \CH^r(X)
    \end{gather*}
  \end{enumerate}
\end{DefLem}

As mentioned above, our next goal is a natural homomorphism of graded rings
\begin{gather*}
  \CL_X\colon \CH^*(X)\rightarrow H^*(X)
\end{gather*}
For its definition we first need to introduce another identification
of cohomology groups involving relative cohomology.

\begin{Def}\label{def:relcoh}
  Let $X$ be a $k$-variety and $Z\subset X$ a closed subvariety.
  We define the relative global section functor as
  \begin{align*}
    \Gamma_Z(X,\argreplacement) \colon \Sh(X) &\longrightarrow \text{AbGrp}\\
    \F &\longmapsto \ker\left(\Gamma(X,\F)\to \Gamma(X\setminus S,\F)\right)
  \end{align*}
  For $r\in\Z$ we call the $r$th right derivative of
  $\Gamma_Z(X,\argreplacement)$ the $r$th relative étale cohomology of the pair
  $(X,Z)$, and write
  $H_Z^r(X,\F)\coloneqq\left(\mathrm R^r\Gamma_Z(X,\argreplacement)\right)(\F)$.
\end{Def}
This is an analogue to relative singular cohomology
$H_{Z,\text{sing}}^r(X,F)=H_{\text{sing}}^r(X,U,F)$
of a pair $(X,U=X\setminus Z)$ of topological spaces with coefficients
in an abelian group $F$. Here one obtains for any pair of topological
spaces a long exact sequence involving relative cohomology.
Even though relative étale cohomology is limited to special pairs of
varieties, it also yields this desirable property:
\begin{Prop}\label{lesrelcoh}
  For a $k$-variety $X$ and a closed subvariety $Z$ of $X$,
  $\F\in\Sh(X)$ there is a long exact sequence of cohomology of the
  pair $(X,Z)$
  \begin{gather*}
    \dotsc
    \to H_Z^r(X,\F)
    \to H^r(X,\F)
    \to H^r(X\setminus Z,\F)
    \to H_Z^{r+1}(X,\F)
    \to \dotsc
  \end{gather*}
\end{Prop}
\begin{Rem}
  There hold even more analogues to the Eilenberg-Steenrod Axioms,
  \forexample excision and additivity \cite[][Chap.~9]{milne}.
\end{Rem}

We will further need the following correspondence between compact and
relative cohomology, which yields---inserted into the long exact
sequence from above---an analogue to the Gysin sequence of a of a
vector bundle in topology.
\begin{Thm}[Gysin isomorphism]\label{gysiniso}
  Let $X$ be a non-singular $k$-variety,
  $Z\subset X$ a closed, non-singular subvariety of codimension $c$,
  $r\in\Z$,
  and $\F\in\Sh(X)$ a locally constant $\Zmod{n}$-module.
  Then there is a canonical and natural isomorphism
  \begin{gather*}
    H^{r}\left(Z,\F\right)
    \xrightarrow{\cong}
    H_Z^{r+2c}\left(X,\F(c)\right)
  \end{gather*}    
\end{Thm}
Mind that we will need non-singularity whenever we want to apply
this isomorphism.
\begin{Def}
  For $X,Z,r,\F$ as in \ref{gysiniso} one can replace
  $H^{r}(Z,\F)$ in the cohomology sequence of the pair
  $(X,Z)$ (see \ref{lesrelcoh}) using the Gysin isomorphism, and thus
  obtains a long exact sequence, called---as in topology---the
  \emph{Gysin sequence} of the pair $(X,Z)$:
  \begin{gather*}
    \dotsc
    \to H_Z^{r-2c}\left(X,\F(-c)\right)
    \to H^r(X,\F)
    \to H^r(X\setminus Z,\F)
    \to H_Z^{r+1}(X,\F)
    \to \dotsc    
  \end{gather*}
  We call the composition
  \begin{gather*}
    H^r(Z,\Lambda)
    \xrightarrow[\text{\ref{gysiniso}}]{\cong}
    H_Z^{r+2c}(X,\Lambda(r))
    \xrightarrow[\text{\ref{lesrelcoh}}]{\text{l.e.s.}}
    H^{r+2c}(X,\Lambda(r))
  \end{gather*}
  the \emph{Gysin map} of the pair $(X,Z)$.
\end{Def}

\begin{DefLem}\label{def:cycleclassmap}
  Let $X$ be a non-singular $k$-variety.
  Then there exists a unique, natural (both in $X$ and coefficients)
  homomorphism of graded rings doubling degree
  \begin{gather*}
    \SwapAboveDisplaySkip
    \CL_X\colon \CH^*(X)\rightarrow H^{2*}(X)\coloneqq
    \bigoplus_{c\in\N} H^{2c}(X,\Lambda(c))
  \end{gather*}
  that fulfills
  \begin{enumerate}
  \item In degree~1 it is the map
    \begin{gather*}
      \SwapAboveDisplaySkip
      \CL_X^1\colon
      \CH^1(X)\cong \Pic(X)\cong H^1(X,\mathds G_m)
      \xrightarrow{\small \substack{\text{Kummer}\\\text{sequence}}}
      H^2(X,\mu_n)
      \overset{\text{Def.}}{\underset{\ref{def:coefficients}}{=}}
      H^2(X,\Lambda(1))
    \end{gather*}
  \item For $Z\in\CH^c(X)$ non-singular, $\CL_X^c(Z)$ is the image of
    $1_{\Zmod{n}}$ under the Gysin map
    \begin{gather*}
      \Zmod{n} \cong H^0(Z,\Lambda)
      \xrightarrow[\text{\ref{gysiniso}}]{\cong}
      H_Z^{2c}(X,\Lambda(r))
      \xrightarrow[\text{\ref{lesrelcoh}}]{\text{l.e.s.}}
      H^{2c}(X,\Lambda(r))
    \end{gather*}
  \end{enumerate}
\end{DefLem}

\begin{Ex}\label{ex:clxx}
  As an important case note that for a non-singular variety $X$
  by definition
  \begin{gather*}
    \CL_{X}^0(X)=\one{X}
  \end{gather*}
  since $X$ is the only irreducible
  subvariety of codimension~$0$ and the Gysin map is an isomorphism in
  this case ($H^0(X\setminus X,\Lambda)=0$ in the Gysin
  sequence).
\end{Ex}

\section{Poincaré Duality}
Again throughout this section $k$ will be an algebraically closed field,
$n\in\Z$ an integer such that $\Char(k)\nmid n$,
and $\Lambda=\Lambda_n$ as in \ref{def:coefficients}.

Remember that in topology, singular cohomology with compact support is
defined as a limit of relative cohomologies by
$H_{\text{c,sing}}(X,\Z)\coloneqq
\dirlim_{Z\subset X \text{cpt.}}H_{Z,\text{sing}}^r(X,\Z)$.
Unfortunately our definition in \ref{def:cptcohomology} had to differ from
this approach as it does not behave well for étale cohomology.
However, we can still recover the following relation:
\begin{DefLem}\label{def:etapoint}
  Let $X$ be a $k$-variety of dimension $d$ and $Z\subset X$ this time
  a complete subvariety. For any torsion sheaf $\F\in\Sh(X)$ and
  $r\in\Z$ we get canonical maps
  \begin{gather*}
    H_Z^r(X,\F)\xrightarrow{\cong} H_c^r(X,\F)
  \end{gather*}
  In degree~0 this yields an isomorphism
  \begin{gather*}
    \dirlim_{\mathclap{Z\subset X \text{cpt.}}} H_Z^0(X,\F) 
    \xrightarrow{\cong} H_c^0(X,\F)\;.
  \end{gather*}
  For a closed point $P\in X$, $\eta_P$ denotes the image of
  $1_\Zmod{n}$ under the Gysin map of the pair $(X,\{P\})$
  \begin{gather*}
    \SwapAboveDisplaySkip
    \Zmod{n}
    \cong H^0(P,\Lambda)
    \xrightarrow[\text{\ref{gysiniso}}]{\cong}
    H_{\{P\}}^{2d}(X,\Lambda(d))
    \longrightarrow H_c^{2d}(X,\Lambda(d))\;.
  \end{gather*}
\end{DefLem}

Now we can state an analogue to Poincaré duality for étale cohomology:
\begin{Thm}[Poincaré]\label{poincare}
  Let $X$ be a non-singular $k$-variety,
  $d\coloneqq\dim(X)$,
  and let $\F\in\Sh(X)$ be a locally constant constructible
  $\Lambda$-module.
  \begin{enumerate}
  \item There is a unique, natural isomophism,
    called the \emph{trace isomorphism},
    \begin{gather*}
      \Tr_X\colon H_c^{2d}(X,\Lambda(d)) \xrightarrow{\cong} \Zmod{n}
      \;,
    \end{gather*}
    such that for any closed point $P\in X$ holds
    $\tr{\eta_P}=1_{\Zmod{n}}$.
  \item There are canonical, natural, perfect pairings of finite groups
    \begin{gather*}
      H_c^r(X,\F) \otimes H^{2d-r}\left(X,\F^\vee(d)\right)
      \overset\cup\longrightarrow H_c^{2d}(X,\Lambda(d))
      \xrightarrow[\Tr_X]{\cong} \Zmod{n}
      \;.
    \end{gather*}
  \end{enumerate}
\end{Thm}

\begin{Def}
  Poincaré duality makes it possible to define a generalised
  intersection number for a non-singular $k$-variety $X$
  of dimension $d$ and $r,s\in\N$ with $r+s=d$:
  \begin{gather*}
    \intNum{\,\cdot\,}\colon
    \CH^r(X)\otimes\CH^s(X)
    \xrightarrow{\intProd{-}{-}} \CH^{d}(X)
    \xrightarrow{\CL_X} H^2d(X,\Lambda(d))
    \xrightarrow[\Tr_X]{\cong} \Zmod{n}
  \end{gather*}
  Passing to the limit and tensoring with $\Ql$ we get
  $\intNum{\,\cdot\,}\colon\CH^r(X)\otimes\CH^s(X)\to\Ql(d)\cong\Ql$
  (see \ref{changeofcoefficients} for further explanation).
\end{Def}


% \begin{Cor}
%   We directly obtain the followin consequences
%   \begin{enumerate}
%   \item 
%   \end{enumerate}
% \end{Cor}

\begin{Def}\label{def:pushforward}
  Let $X$ and $Y$ be smooth $k$-varieties, $r\in\Z$,
  $d\coloneqq\dim(X)$, $d'\coloneqq\dim(Y)$, and $c\coloneqq d'-d$.
  For $\pi\colon X\to Y$ proper we obtain a pushforward morphism on
  cohomology
  \begin{gather*}
    \pi_*\colon
    H^r(X,\Lambda)
    \underset{\ref{poincare}}\cong
    \left(H_c^{2d-r}(X,\Lambda(d))\right)^\vee
    \xrightarrow{(\pi^*)^\vee}
    \left(H_c^{2d-r}(Y,\Lambda(d))\right)^\vee
    \underset{\ref{poincare}}\cong
    H^{r+2c}(Y,\Lambda(c))
    \;.
  \end{gather*}
\end{Def}

\begin{Rem}\label{proppushforward}
  Let $X,Y,d,d',c,r,\pi$ be as in \ref{def:pushforward} above,
  and further let
  $x\in H^r(X,\Lambda)\cong \left(H^{2d-r}(Y,\Lambda(c))\right)^\vee$,
  $y\in H_c^{2d-r}(X,\Lambda)\cong
  \left(H^{r}(X,\Lambda(c))\right)^\vee$.
  Then $\pi_*$ fulfills the following properties:
  \begin{enumerate}[label=(P\arabic*)]
  \item\label{pushforward:altdef}
    $\pi_*$ is uniquely determined by the identity
    \begin{gather*}
      \Tr_{Y}\left(\pi_*(x)\cup y\right) =
      \Tr_{X}\left(x\cup\pi^*(y)\right)
      \;.
    \end{gather*}
  \item\label{pushforward:composition}
    Taking pushforward commutes with composition, \idest for two
    proper morphisms of smooth $k$-varieties
    $\pi_1\colon X\to Y$, $\pi_2\colon Y\to Z$ holds
    \begin{gather*}
      (\pi_2\circ \pi_1)_* = (\pi_2)_*\circ(\pi_1)_*\;.
    \end{gather*}
    This is easily verified by checking \ref{pushforward:altdef}:
    \begin{align*}
      \Tr_X((\pi_2)_*(\pi_1)_*(x) \cup y)
      &\overset{\mathclap{\text{\ref{pushforward:altdef}}}}=
        \Tr_X((\pi_1)_*(x) \cup (\pi_2)^*(y))
      \overset{\mathclap{\text{\ref{pushforward:altdef}}}}=
        \Tr_X(x \cup (\pi_1)^*(\pi_2)^*(y))\\
      &=\Tr_X(x \cup (\pi_2\circ\pi_1)_*(y))
      \overset{\mathclap{\text{\ref{pushforward:altdef}}}}=
        \Tr_X((\pi_2\circ\pi_1)_*(x)\cup y)
    \end{align*}
  \item\label{P3}
    If $\pi$ is a closed immersion, then $\pi_*$ is the Gysin map of
    the pair $(X,Y)$. Especially, in degree~0 we have
    \begin{align*}
      \pi_*\colon
      \Zmod{n}\cong H^{0}(X,\Lambda)
      \xrightarrow[\ref{gysiniso}]{\cong}
      H_X^{2c}(Y,\Lambda(c))
      \xrightarrow[\ref{lesrelcoh}]{\text{l.e.s.}}
      H^{2c}(Y,\Lambda(c))
      \;.
    \end{align*}
    Since $X\cong\pi_*X\in\CH^{c}(Y)$, we get by definition of
    the cycle class map in \ref{def:cycleclassmap}
    \begin{gather*}
      \pi_*(\one{X})=\cl{Y}{\pi_*(X)}\in H^{2c}(Y,\Lambda(c))\;.
    \end{gather*}
  \item\label{pushforward:trace}
    If both $X$ and $Y$ are complete, one obtains for
    $x\in H^{2d}(X,\Lambda(d))$ the equality
    \begin{gather*}
      \Tr_Y(\pi_*(x))
      = \Tr_Y(\pi_*(x)\cup \one{Y})
      \overset{\ref{pushforward:altdef}}= \Tr_X(x\cup\pi^*(\one{Y}))
      = \Tr_X(x)
      \;.
    \end{gather*}
    Mind that $\pi^*(\one{Y})=\one{X}$
    \cite[compare][Rem.~V.11.6.]{milnebook} %% MISSING
  \item
    For $X$ and $Y$ complete furthermore holds
    \begin{gather}
      \tag{Projection Formula}\label{pushforward:projectionformula}
      \pi_*(x\cup \pi^*(y)) = \pi_*(x)\cup y
    \end{gather}
    where this time $x\in H^r(X,\Lambda)$ and $y\in H^r(Y,\Lambda)$
    for arbitrary $r,s\in\Z$. This is again easily shown by checking
    \ref{pushforward:altdef}:
    \begin{align*}
      \Tr_X\big((\pi_*(x)\cup y) \cup z\big)
      &\overset{\mathclap{\text{\ref{pushforward:altdef}}}}=
      \Tr_X\big(\pi_*(x)\cup (y \cup z)\big)
      \overset{\mathclap{\text{\ref{pushforward:altdef}}}}=
      \Tr_X\big( x\cup \pi^*(y\cup z)\big)\\
      &\overset{\mathclap{\text{\ref{pushforward:altdef}}}}=
      \Tr_X\big( (x\cup \pi^*(y)) \cup \pi^*(z) \big)
      \overset{\mathclap{\text{\ref{pushforward:altdef}}}}=
      \Tr_X\big( \pi_*(x\cup \pi^*(y)) \cup z\big)
    \end{align*}
  \end{enumerate}
\end{Rem}

\section{Lefschetz Fixed-Point Formula}

\begin{Thm}[Lefschetz fixed-point formula]\label{lefschetzthm}
  Let $X$ be a non-singular, complete variety over an algebraically
  closed field $k$.
  Let further $\phi\colon X\to X$ be a regular Endomorphism.
  Then
  \begin{gather*}
    \SwapAboveDisplaySkip
    \intNum{\intProd{\Graph{\phi}}{\Diag{X}}}
    = \sum_{r\in\N} (-1)^r \trace{\phi^*}{H^r(X,\Ql)}
  \end{gather*}
  where $\trace{\phi^*}{H^r(X,\Ql)}$ is the trace of $\phi^*$ as
  Endomorphism of a $\Ql$-vector space. 
  The sum is finite due to the vanishing
  of cohomology in degree greater than $2\dim(X)$. 
\end{Thm}
Note that $\intNum{\intProd{\Graph{\phi}}{\Diag{X}}}$ is the number of
fixed points of $\phi$, hence the name of this formula.


\begin{Rem}\label{changeofcoefficients}
  We first neet to take a look at the coefficients.
  All the maps and identities we have seen so far are natural,
  and hence compatible with taking a direct limit,
  as well as with tensoring with $\Ql$.
  The needed maps and isomorphism are namely
  the cup product,
  the trace isomorphism,
  Poincaré duality,
  the cycle class map,
  pushforwards and pullbacks on cohomology, as well as
  the Künneth isomorphism.
  Using  \ref{cohomologyoftwists} we get with
  $\Ql(r)\coloneqq\left(\lim_{n\to\infty}\Zmod{l^n}(r)\right)\otimes\Ql$
  for $r\in\N,s\in\Z$ the identity
  \begin{align*}
    H^r(X,\Ql(s)) &=
    \left(\lim_{n\to\infty} H^r(X,\M_{l^n}(s))\right) \otimes \Ql\\
    &\cong
    \left(\lim_{n\to\infty} H^r(X,\M_{l^n})\otimes \Zmod{l^n}(s)\right)\otimes\Ql
    \eqqcolon H^r(X,\Ql)\otimes\Ql(s)
  \end{align*}
  As $\Ql(s)$ is a 1-dimensional $\Ql$-vectorspace and $\phi^*$ only
  acts through $H^r(X,\Ql)$, we will work with the $\Ql$-vector spaces
  $H^*(X)\coloneqq\sum_{r\in\N} H^r(X, \Ql(r))$. This then
  already implies the theorem
  since $\trace{\phi^*}{H^r(X)}=\trace{\phi^*}{H^r(X,\Ql)}$
  \cite[][Rem.~I.25.5]{milne}.
\end{Rem}

\begin{Rev}
  Let $V$ be a finite dimensional $K$-vector space, $(e_i)_{i=1}^n$ a basis,
  and $(\dual e_i)_{i=1}^n$ the corresponding dual basis.
  Denote the evaluation by
  \begin{align*}
    \cup\colon V\otimes V^* \to K,\quad
    v\otimes \omega \mapsto v\cup\omega\coloneqq \omega(v)
  \end{align*}
  Then the trace of some $f\in\End{V}$ is defined as
  \begin{gather*}
    \trace{f}{V} \coloneqq \sum_{i=1}^n f(e_i)\cup \dual e_i
  \end{gather*}
\end{Rev}

  %\begin{enumerate}[label={Step \arabic*.}]
    %% STEP % [25.3]
    First we need the following identity:
    \begin{Lem}\label{step1}
      For a regular morphism $\phi\colon X\to Y$ of complete varieties
      and $y\in H^*(Y)$ holds
      \begin{gather*}
        \SwapAboveDisplaySkip
        p_*\left(\cl{X\times Y}{\Graph{\phi}} \cup q^*(y)\right)
        = \phi^*(y)           
      \end{gather*}
      \begin{proof}
        Since $Y$ is separated, $\Graph{\phi}\colon X\to X\times_k Y$
        is a closed immersion, hence proper, and we can apply the
        \ref{pushforward:projectionformula}.
        Furthermore, by naturality of the cycle class map we have
        \begin{gather}
          \SwapAboveDisplaySkip
          \cl{X\times_k Y}{\Graph{\phi}} \coloneqq
          \cl{X\times_k Y}{(\Graph{\phi})_*(X)} =
          (\Graph{\phi})_*(\cl{X}{X}) \overset{\ref{ex:clxx}}{=} % REF MISSING          
          (\Graph{\phi})_*\left(\one{X}\right)
        \end{gather}
        Altogether we get
        \begin{align*}
          p_*\left(\cl{X\times Y}{\Graph{\phi}} \cup q^*(y)\right)
          &=  p_*\left(
            (\Graph{\phi})_*\left(\one{X}\right) \cup q^*(y)
            \right)\\
          &\overset{\mathllap{\text{\ref{pushforward:projectionformula}}}}{=}
            p_*(\Graph{\phi})_*\left( \one{X} \cup (\Graph{\phi})^*q^*(y) \right)\\
          &\overset{\mathllap{\ref{pushforward:composition}}}=
            (p\circ\Graph{\phi})_* \left(
            \one{X} \cup (q\circ\Graph{\phi})^*(y) \right)\\
          &\overset{\mathllap{\text{Def. $\Graph\phi$}}}=
            \id_* \left( \one{X} \cup \phi^*(y) \right)\\
          &= \phi^*(y)
        \end{align*}
      \end{proof}
    \end{Lem}

    %% STEP
    Now we can explicitly describe
    $\cl{X\times_k X}{\Graph{\phi}}$:
    \begin{Lem}\label{clofgraph}
      Let $X$ be a complete, non-singular variety of dimension $d$
      and $\altphi$ a regular Endomorphism of $X$.
      Write $e^{2d}\in H^{2d}(X)$ for the canonical generator
      with $\tr{e^{2d}}=1\in\Ql$.
      Furthermore, let $(b_k)_k$ be a basis of $H^*(X)$ as
      $\Ql$-vector space and $(f_k)_k$ be the dual Basis with respect
      to the cup product, \idest $b_k\cup f_l=\delta_{kl}e^{2d}$
      (exists by \nameref{poincare}).
      Then
      \begin{gather*}
        \SwapAboveDisplaySkip
        \cl{X\times_k X}{\Graph{\phi}} =
        \sum_{k} \phi^*(b_k)\otimes  f_k
      \end{gather*}
      \begin{proof}
        By the \nameref{kuennethiso} % REF MISSING
        $H^*(X\otimes X) \cong H^*(X)\otimes_{\Ql}H^*(X)$ is a
        $H^*(X)$-module with basis
        $(\one{X}\otimes f_k)_{k}$.
        Therefore, there are unique factors $a_{k}$ in $H^*(X)$ s.t.
        \begin{gather}
          \label{eq:step2}
          \cl{X\times_k Y}{\Graph{\phi}} =
          \sum_{k} a_k\otimes f_k
        \end{gather}
        Explicitly
        \begin{align*}
          \SwapAboveDisplaySkip
          a_j
          = p_*(a_j\otimes e^{2d})
          &= p_*\Big(
            \sum_k(a_k\otimes f_k)\cup (\one{X}\otimes b_j)
            \Big)\\
          &\overset{\mathllap{\eqref{eq:step2}}}= p_*\left(
            \cl{X\times_k X}{\Graph{\phi}} \cup q^*(b_j)
            \right)\\
          &\overset{\mathllap{\ref{step1}}}= \phi^*(b_j) 
        \end{align*}
      \end{proof}
    \end{Lem}

    %% STEP
    \begin{proof}[proof of \ref{lefschetzthm}]~
    For all $r\in\{0,\dotsc,2d\}$ let $(e_i^r)_{i}$ be a basis of
    $H^r(X)$ as $\Ql$-vector space. A union of these obviously yields
    a basis of $H^*(X)$ as $\Ql$-vector space, whose Poincaré dual
    basis will be the union of degreewise dual bases
    $(\dual e_i^{2d-r})_{i}\subset H^{2d-r}(X)\cong (H^{r}(X))^\vee$,
    \idest
    $e^r_i\cup \dual e_j^{2d-s}=\delta_{ij}\delta_{rs}e^{2d}$.
    Applying Lemma~\ref{clofgraph} with bases $(e_i^r)_{i,r}$ and
    $(\dual e_i^{2d-r})_{i,r}$ to $\phi^*$ and $\id_X^*$ yields
    \begin{align*}
      \cl{X\times_k X}{\Graph{\phi}}
      &= \sum_{i,r} \phi^*(e_i^r)\otimes \dual e_i^{2d-r}\\
      \cl{X\times_k X}{\Diag{X}}
      &= \sum_{i,r} e_i^r\otimes \dual e_i^{2d-r} =
        \sum_{i,r} (-1)^{r\cdot (2d-r)}\dual e_i^{2d-r}\otimes e_i^{r} =
        \sum_{i,r} (-1)^{r}\dual e_i^{2d-r}\otimes e_i^r\\
      \cl{X\times_k X}{\intProd{\Graph{\phi}}{\Diag{X}}}
      &= \cl{X\times_k X}{\Diag{X}} \cup \cl{X\times_k X}{\Graph{\phi}}\\
      &= \sum_{i,r,j,s} (-1)^{r}
        \left(\phi^*(e_i^r)\cup \dual e_j^{2d-s}\right)
        \otimes \left(\dual e_i^{2d-r}\cup e_j^s\right)\\
      &= \sum_{i,r} (-1)^{r}
        \left(\phi^*(e_i^r)\cup \dual e_i^{2d-r}\right)
        \otimes e^{2d}\\
      &\overset{\mathllap{\Poincare}}=
        \sum_{r} (-1)^r \trace{\phi^*}{H^r(X)} e^{2d} \otimes e^{2d}
    \end{align*}
    Altogether, using
    \begin{align*}
      \SwapAboveDisplaySkip
      \tr{e^{2d} \otimes e^{2d}}
      = \tr{\Diag{X}_*(e^{2d})}
      \overset{\ref{pushforward:trace}}= \tr{e^{2d}}
      \overset{\text{Def.}}= 1\in\Ql
      \;,
    \end{align*}
    we get our final result:
    \begin{gather*}
      \intNum{\intProd{\Graph\phi}{\Diag X}} \coloneqq
      \tr{\cl{X\times_kX}{\intProd{\Graph\phi}{\Diag X}}}
      = \sum_{r\in\N} (-1)^r \trace{\phi^*}{H^r(X,\Ql)}
    \end{gather*}
%  \end{enumerate}
\end{proof}

\nocite{*}
\printbibliography
\end{document}
