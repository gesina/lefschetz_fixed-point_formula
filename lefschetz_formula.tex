\documentclass[english]{scrartcl}
\usepackage{fontspec}
\usepackage{babel}
\usepackage[backend=biber]{biblatex}
\bibliography{lefschetz_formula.bib}
\usepackage{scrlayer-scrpage}
\usepackage{mathtools, amssymb, amsthm, dsfont}
\usepackage{enumitem}
\setlist[description]{font=\normalfont\bfseries}
\usepackage{tikz-cd}

\usepackage{hyperref}
\hypersetup{
  pdfauthor=Gesina Schwalbe,
  pdftitle=Lefschetz fixed-point formula
}

% THEOREMS
\theoremstyle{definition}
\newtheorem{Def}{Definition}[section]
\newtheorem{DefLem}[Def]{Definition/Lemma}
\newtheorem{Prop}[Def]{Proposition}
\newtheorem{Thm}[Def]{Theorem}
\newtheorem{Lem}[Def]{Lemma}
\theoremstyle{remark}
\newtheorem{Rem}[Def]{Remark}
\newtheorem{Rev}[Def]{Reminder}

% SHORTENINGS
\newcommand*{\N}{\mathds{N}}
\newcommand*{\Z}{\mathds{Z}}
\newcommand*{\Q}{\mathds{Q}}
\newcommand*{\R}{\mathds{R}}
\newcommand*{\C}{\mathds{C}}
\newcommand*{\fF}{\mathds{F}} % finite field
\newcommand*{\Zmod}[1]{\Z/#1} % Z/nZ
\newcommand*{\Zl}{\Z_l} % l-adic completion of Z
\newcommand*{\Ql}{\Q_l} % quotient field of \Zl
\newcommand*{\F}{\mathcal{F}} % sheaf F
\newcommand*{\G}{\mathcal{G}} % sheaf G
\newcommand*{\End}[1]{\text{End}(#1)} % Endomorphisms
\DeclareMathOperator{\Ext}{Ext} % Ext functor
\DeclareMathOperator{\Hom}{Hom} % Hom functor
\newcommand*{\id}{\text{id}} % identity
\DeclareMathOperator{\Sh}{Sh} % sheaves over a space
\newcommand*{\idest}{i.e.\ }
\newcommand*{\forexample}{e.g.\ }

\newcommand*{\M}{\Lambda}
% \M_l(r)(R) = {
% for r=0 \ZlZ;
% for r>0 \mu_l(R)^{\otimes r};
% for r<0 Hom_{\Zmod{l}}(\mu_l(R)^{\otimes -r}, \Zmod{l}) } 
\newcommand*{\tr}[1]{\text{tr}\left(#1\right)} % trace isomorphism
\newcommand*{\intProd}[2]{{#1\cdot#2}} % intersec. product in the Chow ring
\newcommand*{\intNum}[1]{{\left\langle{#1}\right\rangle}} % intersection number (= tr(intersection product))
\newcommand*{\Graph}[1]{{\Gamma_{#1}}} % graph of a map
\newcommand*{\Diag}[1]{{\Delta_{#1}}} % diagonal (both map and subscheme)
\newcommand*{\trace}[2]{{\text{trace}\left(#1 \,\middle|\, #2 \right)}} % trace of map (#1) in vector space (#2)
\newcommand*{\cl}[2]{{\text{cl}_{#1}\left(#2\right)}} % cycle class map
\newcommand*{\one}[1]{{1_{#1}}}%{H^*(#1)}}
\newcommand*{\dual}[2]{{#1_{#2}^*}} % dual basis vector
\let\altphi\phi
\renewcommand*{\phi}{\varphi}
\newcommand*{\Poincare}{{\text{Poincaré \ref{poincare}}}}
\newcommand*{\Hc}{{H_c}} % cohomology with compact support

\begin{document}
% HEAD/FOOT
\clearpairofpagestyles
\ohead{Universität Regensburg}
\ihead{Gesina Schwalbe}
\cfoot*{\pagemark}

% TITLE
\title{Poincaré Duality and\\
  Lefschetz Fixed-Foint Formula}
\subject{Seminar:
  Deligne's proof of the Weil Conjecture%\\by Federico Binda, Prof. Dr. Moritz Kerz
}
%\subtitle{Lecture Notes}
\author{Gesina Schwalbe}
%\date{}
\maketitle
\tableofcontents

\section{Motivation}
From topology the following theorem is known that relates the fixed
points of a morphism (counted with multiplicity) with homology:
\begin{Thm}[Lefschetz-Hopf]
  Let $X$ be a compact, smooth manifold and
  $\phi\colon X\to X$ a morphism. Then
  \begin{gather*}
    \sum_{x\in\operatorname{Fix}(\phi)} \operatorname{i}(\phi,x)
    =\sum_{r\in\N}(-1)^r \trace{\phi^*}{H_\text{sing}^r(X,\Q)}\;,
  \end{gather*}
  where $\operatorname{Fix}(\phi)$ are the fixed points of $\phi$ and
  $\mathrm i(\phi,x)$ is the multiplicity of a fixed point $x$ of $\phi$
  in $X$.
\end{Thm}
The proof uses the Poincaré duality theorem which says
\begin{Thm}[Poincaré]
  For a smooth manifold $X$ holds for all $r\in\Z$
  \begin{gather*}
    H_{\text{c,sing}}^r(X,\Z) \cong H_{\dim(X)-r}^{\text{sing}}(X,\Z)
  \end{gather*}
  where $H_{\text{c,sing}}(X,\Z)\coloneqq
  \lim_{Z\subset X \text{cpt.}}H_{Z,\text{sing}}^r(X,\Z)$
  is cohomology with compact support for $X$
  (write $H_{Z,\text{sing}}^r(X,\Z)$ for $r$th relative
  $\Z$-cohomology of the pair $(X,Z)$).
\end{Thm}
The aim of this lecture is to introduce an analogue statement for
schemes instead of topological spaces as well as give a proof and an
overview of the necessary preliminaries.
Some convenient correspondences are
\begin{itemize}
\item an algebraically closed field $k$ instead of $\R$ as ground field,
\item $k$-\emph{variety}
  (\idest separated, geometrically reduced scheme of finite type over $k$)
  instead of \emph{$\R$-manifold}
\item \emph{complete} instead of \emph{compact},
\item \emph{non-singular} instead of \emph{smooth}
\item \emph{morphism of varieties} instead of \emph{morphism of manifolds}
\item \emph{étale cohomology} instead of \emph{singular homology}
  (mind that they coincide for non-singular varieties over $\C$ and
  finite coefficient group \cite[see][Thm.~21.1]{milne})
\item intersection number $\intNum{\intProd{\Graph{\phi}}{\Diag{X}}}$
  (also counting fixed points of $\phi$ with multiplicity)
  instead of $\sum_{x\in\operatorname{Fix}(\phi)}\mathrm{i}(\phi,\cdot)$.
\end{itemize}
We also need to consider modified coefficients since étale cohomology
does not behave well for non-finite coefficient groups. However,
passing to limits we can consider for $X$ (étale) scheme the étale
cohomology with coefficients in $\Ql$ as follows
\begin{gather*}
  H^r(X,\Ql)\coloneqq
  \left(\lim_{n\to\infty}H^r(X,\Zmod{l^n})\right)\otimes_{\Zl}\Ql
\end{gather*}
Thus the final statement of the Lefschetz fixed point formula shall be
\begin{Thm}[Lefschetz]\label{lefschetzthm:motivation}
  Let $k$ be an algebraically closed field, $X$ a non-singular,
  complete $k$-variety, and let $\phi\colon X\to X$ be an
  endomorphism. Then
  \begin{gather*}
    \intNum{\intProd{\Graph{\phi}}{\Diag{X}}}
    = \sum_{r\in\N}(-1)^r\trace{\phi^*}{H^r(X,\Ql)}
  \end{gather*}
\end{Thm}

We instantly get the following two applications:
\begin{description}
\item[Zeta function] For
  $k=\fF_{q}$ a finite field ($q\in\N$),
  $X$ a non-singular variety over $k$ and
  $\phi=F^n$ the $n$th Frobenius homomorphism
  we get
  \begin{gather*}
    \#X(\fF_{q^n}) = \intNum{\intProd{\Graph\phi}{\Diag X}}
  \end{gather*}
  which is essential to further investigate the Zeta function of $X$.
\item[Euler characteristic] Taking $k$ and $X$ as in the Theorem and
  $\phi=\Delta$ we see that the étale Euler characteristic of $X$ is
  just its self-intersection number, \idest
  \begin{gather*}
    \intNum{\intProd{\Delta}{\Delta}} = \chi_{\text{ét}}(X)
  \end{gather*}
\end{description}

For the proof of \ref{lefschetzthm:motivation}
some further nice properties of étale cohomology for
(nice) varieties will be needed, such as:
a Künneth isomorphism (\autoref{kuennethiso}),
a cycle class map that gives a connection to the Chow ring and
intersection product (\autoref{cycleclassmap}),
Poincaré duality with a trace map (\autoref{poincare}),
as well as pushforwards on cohomology (\autoref{gysinmaps}).
To establish these, one needs for étale cohomology the notion of
cohomology with compact support (\autoref{compactcoh}),
relative cohomology (\autoref{relcoh}),
long exact sequences of (compact) pairs (\autoref{LES}),
Gysin sequences (\autoref{gysinsequ}),
and most of all a cup product, \idest a graded ring structure
(\autoref{cupproduct}).
All of this will be introduced in the following.

\section{Preliminaries on étale Cohomology}
It will be assumed the reader knows about the definition and basic
properties of étale cohomology. If not stated otherwise, for a
(Zariski) scheme $X$ and a (Zariski) sheaf of modules $\F$ over $X$ we
denote with $H^r(X,\F)$ the $r$th étale cohomology group of the étale
sheaf $\F$ over the étale scheme of $X$
\cite[compare][Chap.~6, Examples of Sheaves on $X_\text{ét}$]{milne}.

Also recall that for $\F$ torsion sheaf or $X$ complete these groups
are finite \cite[\forexample][Thm.~19.1]{milne}.

\subsection{Étale Cohomology with compact support}
In this section let $k$ be an algebraically closed field,
$U$ a variety over $k$,
and $\F$ be a torsion sheaf over $U$.

The definition of cohomology with compact support as
$\left(\lim_{Z\subset X \text{cpt.}}H_Z^r(X,\Z)\right)$
cannot be copied from topology but has to be adjusted.
However, we will see later (namely \autoref{rem:cptvsrelcohomology}),
that at least in degree 0 the following definition
and the expression above coincide.

First recall the following
\begin{Rev}[Nagata compactification]\label{nagata}
  There always exists a complete $k$-variety $X$ and an open, dense
  immersion $j\colon U\to X$. $j$ is also called a
  \emph{compactification} of $U$.
\end{Rev}
\begin{Rev}[Extension by zero]\label{def:extbyzero}
  For an open immersion $j\colon U\to X$ into
  a scheme $X$, $j_!$ (called the \emph{extension by zero} of $j$)
  is defined as
  \begin{gather*}
    j_!\colon \Sh(U)\to \Sh(X)\;,\qquad
    j_!\F \coloneqq \text{sheafification of}\quad
    V\mapsto\begin{cases}
      \F(V) & V\subset U\\
      0     & \text{else}
    \end{cases}
  \end{gather*}
\end{Rev}

\begin{DefLem}\label{def:cptcohomology}
  Let $j\colon U\to X$ be a compactification of $U$. Then the
  cohomology with compact support in degree $r\in\Z$ is defined as
  \begin{gather*}
    \Hc^r(U,\F) \coloneqq H^r(X,j_!(\F))
  \end{gather*}
  This definition is independent of the choice of compactification.
  \begin{proof}[proof (scetch)]
    The set of compactifications of $U$ is partially ordered by proper
    morphisms:
    For two compactifications $j_1\colon U\to X_1$,
    $j_2\colon U\to X_2$ the closure
    $X\coloneqq\overline{(j_1,j_2)(U)}$ in $X_1\times_X X_2$ is again
    a compactification and the projections on $X_1$, $X_2$ are proper.
    Thus one only needs to check $\Hc^r(X,j_!\F)\cong\Hc^r(X',j_!'\F)$
    for compactifications $j$, $j'$ and a proper map $\pi$ such that
    the following diagramm commutes
    \begin{center}
      \begin{tikzcd}
        U\arrow[r,"j"]\arrow[dr,"j'"{left}, bend right=20]
        & X\arrow[d,"\pi"]\\
        & X'
      \end{tikzcd}
    \end{center}
    This then can be shown using a version of the Leray spectral
    sequence and the proper base change theorem.
  \end{proof}
\end{DefLem}

\begin{Rem}
Note that this is no derived functor!
However, since $j_!$ is exact for a compactification $j$ of $U$, we
still get a long exact sequence of cohomology out of a short exact
sequence of a short exact sequence of sheaves on $U$
(just use the definition and the long exact sequence of cohomology for
$H^*(X,\cdot)$).
\end{Rem}

\subsection{Ext-Pairings, Cup Product and Künneth formula}
The foundations of all further investigations now will be the graded
ring structure on the sum of the cohomology groups in all degrees
(with appropriate coefficients).
We here give a general approach to the definition via Ext-groups,
however there can be given more concrete descriptions using
Čech-cohomology \cite[see][Chap.~10 and Chap.~22, Cup-products]{milne}
or later using the Künneth isomorphism (see \autoref{kuennethiso},
\autoref{rem:cupwithkuennethiso}).

\begin{Rem}[Yoneda product]\label{yonedaproduct}
  For an abelian category $\mathcal{A}$ with enough injectives and
  any objects $A,B,C$ of $\mathcal{A}$, $r,s\in\N$ there exists a pairing
  \begin{align*}
    \Ext^r(A,B)\otimes\Ext^s(B,C) &\overset{\cup}{\longrightarrow}
    \Ext^{r+s}(A,C)
    % \intertext{which is induced by}
    \\
    \text{induced by}\qquad
    \Hom(A,B)\otimes\Hom(B,c) &\longrightarrow \Hom(A,C)\;,\quad
                                f\otimes g\mapsto f\circ g
  \end{align*}
  \begin{proof}[idea for construction]
    The elements of an $r$th Ext-group can be described as extensions
    (\idest sequences) of length $r$.
    Define $\cup$ to be the concatenation of two such sequences. 
  \end{proof}
\end{Rem}

\begin{Prop}[Cup product]\label{def:cupproduct}
  For a variety $X$, and $\F,\G\in\Sh(X)$ get for all $r,s\in\Z$
  \begin{gather*}
    H^r(X,\G)\otimes H^r(X,\F) \overset{\cup}\longrightarrow
    H^{r+s}(X,\F\otimes\G)
  \end{gather*}
  \begin{proof}[idea for construction]
    This is exactly the Yoneda product in \autoref{yonedaproduct} with
    respect to the following identities
    \begin{align*}
      H^r(X,\G)&=\Ext_X^r(\mathcal O_X,\G)\\
      H^r(X,\F)&=\Ext_X^r(\mathcal O_X,\F)=\Ext_X^r(\G,\G\otimes\F)\\
      H^r(X,\F\otimes\G)&=\Ext_X^r(\mathcal O_X,\G\otimes\F)
    \end{align*}
  \end{proof}

\end{Prop}

\subsection{The Chow Ring and the Cycle Map}

\section{Poincaré Duality}
\subsection{Trace Map}
\subsection{Poincaré Duality}
\begin{Thm}[Poincaré]
  
\end{Thm}
\subsection{Gysin Maps}

\section{Lefschetz Fixed-Point Formula}

\begin{Thm}[Lefschetz fixed-point formula]\label{lefschetzthm}
  Let $X$ be a non-singular, complete variety over an algebraicly
  closed field $k$.
  Let further $\phi\colon X\to X$ be a regular Endomorphism.
  Then
  \begin{gather*}
    \intNum{\intProd{\Graph{\phi}}{\Diag{X}}}
    = \sum_{r\in\N} (-1)^r \trace{\phi^*}{H^r(X,\Ql)}
  \end{gather*}
  where $\trace{\phi^*}{H^r(X,\Ql)}$ is the trace of $\phi^*$ as
  Endomorphism of a $\Ql$-vector space. 
  The sum is finite due to the vanishing
  of cohomology in degree greater than $2\dim(X)$. 
\end{Thm}
Note that $\intNum{\intProd{\Graph{\phi}}{\Diag{X}}}$ is exactly the number of
fixed points of $\phi$, hence the name of this formula.


\subsection{Proof}
\begin{Rem}
  In the following we will work with
  $H^*(X)\coloneqq\sum_{r\in\N} H^r(X, \Ql(r))$, where
  \begin{gather*}
    H^r(X) = H^r(X,\Ql(r)) \coloneqq
    \left(\lim_{n\to\infty} H^r(X,\M_{l^n}(r))\right) \otimes \Ql
  \end{gather*}
  All the maps and identities we have seen so far are compatible with
  taking the limit and tensoring with $\Ql$, namely
  cup product,
  trace isomorphism,
  Poincaré duality,
  cycle class map and hence
  pushforward and pullback on cohomology as well as
  the Künneth isomorphism.
  Thus $H^*(X)$ together with the induced cup product is a graded ring
  fulfilling these properties.
  For $X$, $\phi$ as above we will deduce the formula
  \begin{gather*}
    \intNum{\intProd{\Graph{\phi}}{\Diag{X}}}
    = \sum_{r\in\N} (-1)^r \trace{\phi^*}{H^r(X)}
  \end{gather*}
  This already implies the theorem
  (\idest $\trace{\phi^*}{H^r(X)})=\trace{\phi^*}{H^r(X,\Ql)}$),
  because:
  Since $k$ is algebraicly closed---especially it contains all
  roots of unity---we can choose for $r=1$ (hence inductively all
  $r\in\N$) and $n\in\N$ an isomorphism
  \begin{gather*}
    \M_{l^n}(r)\coloneqq\mu_{l^n}^{\otimes_k r} \cong \Zmod{(l^n)}
  \end{gather*}
  So $H^r(X)\cong H^r(X,\Ql)$ as $\Ql$-vector spaces.
  One now has to observe, that the traces of $\phi^*$ commute with
  this isomorphism.
  % TODO!!
\end{Rem} 

\begin{Rev}
  Let $V$ be a finite dimensional $K$-vector space, $(e_i)_{i=1}^n$ a basis,
  and $(\check e_i)_{i=1}^n$ the corresponding dual basis.
  Denote the evaluation by
  \begin{align*}
    \cup\colon V\otimes V^* \to K,\quad
    v\otimes \omega \mapsto v\cup\omega\coloneqq \omega(v)
  \end{align*}
  Then the trace of some $f\in\End(V)$ is defined as
  \begin{gather*}
    \trace{f}{V} \coloneqq \sum_{i=1}^n f(e_i)\cup \check e_i
  \end{gather*}
\end{Rev}

\begin{proof}[proof of \autoref{lefschetzthm}]~
  \begin{enumerate}[label={Step \arabic*.}]
  \item %[25.3]
    First we need the following identity:
    \begin{Lem}\label{step1}
      For a regular morphism $\phi\colon X\to Y$ of complete varieties
      and $y\in H^*(Y)$ holds
      \begin{gather*}
        p_*\left(\cl{X\times Y}{\Graph{\phi}} \cup q^*(y)\right)
        = \phi^*(y)           
      \end{gather*}
      \begin{proof}
        Since $Y$ is separated, $\Graph{\phi}\colon X\to X\times_k Y$
        is a closed immersion, hence proper, and we can apply the 
        \nameref{projectionformula}. % REF MISSING
        Furthermore, by functoriality of the cycle class map we have
        \begin{gather}
          \cl{X\times_k Y}{\Graph{\phi}} \coloneqq
          \cl{X\times_k Y}{(\Graph{\phi})_*(X)} =
          (\Graph{\phi})_*(\cl{X}{X}) \overset{\ref{ex:clxx}}{=} % REF MISSING          
          (\Graph{\phi})_*\left(\one{X}\right)
        \end{gather}
        Altogether we get
        \begin{align*}
          p_*\left(\cl{X\times Y}{\Graph{\phi}} \cup q^*(y)\right)
          &=  p_*\left(
            (\Graph{\phi})_*\left(\one{X}\right) \cup q^*(y)
            \right)\\
          &\overset{\mathllap{\nameref{pushforward:projectionformula}}}{=}
            p_*(\Graph{\phi})_*\left( \one{X} \cup (\Graph{\phi})^*q^*(y) \right)\\
          &\overset{\mathllap{\eqref{pushforward:composition}}}=
            (p\circ\Graph{\phi})_* \left(
            \one{X} \cup (q\circ\Graph{\phi})^*(y) \right)\\
          &\overset{\mathllap{\text{Def. $\Graph\phi$}}}=
            \id_* \left( \one{X} \cup \phi^*(y) \right)\\
          &= \phi^*(y)
        \end{align*}
      \end{proof}
    \end{Lem}
  \item Now we can explicitly describe
    $\cl{X\times_k X}{\Graph{\phi}}$:
    \begin{Lem}
      Let $X$ be a complete and non-singular variety of dimension $d$
      and $\altphi\in\End(X)$ regular.
      Write $e^{2d}\in H^{2d}(X)$ for the canonical generator
      with $\tr{e^{2d}}=1\in\Ql$.
      Furthermore, let $(b_k)_k$ be a basis of $H^*(X)$ as
      $\Ql$-vector space and $(f_k)_k$ be the dual Basis with respect
      to the cup product, \idest $b_k\cup f_l=\delta_{kl}e^{2d}$.
      % ?? HOW IS THIS CONSTRUCTED? / WHY DOES THIS EXIST?
      Then
      \begin{gather*}
        \cl{X\times_k X}{\Graph{\phi}} =
        \sum_{k}^{2d} \phi^*(b_k)\otimes  f_k
      \end{gather*}
      \begin{proof}
        By the \nameref{künnethiso} % REF MISSING
        $H^*(X\otimes X) \cong H^*(X)\otimes_{\Ql}H^*(X)$ is a
        $H^*(X)$-module with basis
        $(\one{X}\otimes f_k)_{k}$.
        Therefore, there are unique factors $a_{k}$ in $H^*(X)$ s.t.
        \begin{gather}
          \label{eq:step2}
        \cl{X\times_k Y}{\Graph{\phi}} =
        \sum_{k} a_k\otimes f_k
      \end{gather}
      Explicitly
      \begin{align*}
        a_j
        = p_*(a_j\otimes e^{2d})
        &= p_*\Big(
          \sum_k(a_k\otimes f_k)\cup (\one{X}\otimes b_j)
        \Big)\\
        &\overset{\mathllap{\eqref{eq:step2}}}= p_*\left(
          \cl{X\times_k X}{\Graph{\phi}} \cup q^*(b_j)
          \right)\\
        &\overset{\mathllap{\ref{step1}}}= \phi^*(b_j) 
      \end{align*}
      \end{proof}
    \end{Lem}
  \item
  For all $r\in\{0,\dotsc,2d\}$ let
  \begin{itemize}
  \item $(e_i^r)_{i}$ be a basis of $H^r(X)$ as $\Ql$-vector space,
  \item $(\check e_i^{2d-r})_{i}$ be the dual basis of $H^{2d-r}(X)\cong (H^{r}(X))^\vee$,
    \idest $e^r_i\cup \check e_j^{2d-r}= \delta_{ij}e^{2d}$.
  \end{itemize}
  Note that then $e^r_i\cup \check e_j^{2d-s}=\delta_{ij}\delta_{rs}e^{2d}$,
  % ?? WHY?
  \idest they fulfill the properties of the Lemma
  and we get for $\phi$ and $\id$
  \begin{align*}
    \cl{X\times_k X}{\Graph{\phi}}
    &= \sum_{i,r} \phi^*(e_i^r)\otimes \check e_i^{2d-r}\\
    \cl{X\times_k X}{\Diag{X}}
    &= \sum_{i,r} e_i^r\otimes \check e_i^{2d-r} =
      \sum_{i,r} (-1)^{r\cdot (2d-r)}\check e_i^r\otimes e_i^{2d-r} =
      \sum_{i,r} (-1)^{r}\check e_i^{2d-r}\otimes e_i^r\\
    \cl{X\times_k X}{\intProd{\Graph{\phi}}{\Diag{X}}}
    &= \cl{X\times_k X}{\Diag{X}} \cup \cl{X\times_k X}{\Graph{\phi}}\\
    &= \sum_{i,r,j,s} (-1)^{r}
      \left(\phi^*(e_i^r)\cup \check e_j^{2d-s}\right)
      \otimes \left(\check e_i^{2d-r}\cup e_j^s\right)\\
    &= \sum_{i,r} (-1)^{r}
      \left(\phi^*(e_i^r)\cup \check e_i^{2d-r}\right)
      \otimes e^{2d}\\
    &\overset{\mathllap{\Poincare}}=
      \sum_{r} (-1)^r \trace{\phi^*}{H^r(X)} e^{2d} \otimes e^{2d}
  \end{align*}
  Altogether, using
  \begin{align*}
    \tr{e^{2d} \otimes e^{2d}}
    % TODO!! ??
    % &= \tr{p^*(e^{2d})\cup q^*(e^{2d})}\\
    % &= \tr{p^*(e^{2d})}\cdot\tr{ q^*(e^{2d})}\\
    % &\overset{\eqref{pushforward:trace}=
    %   \tr{e^{2d}}\cdot\tr{e^{2d}}\\
    %   \overset{\text{Def.}}= 1\cdot 1
    = \tr{\Diag{X}_*(e^{2d})}
    \overset{\eqref{pushforward:trace}}= \tr{e^{2d}}
    \overset{\text{Def.}}= 1\in\Ql
  \end{align*}
  we get our final result:
  \begin{gather*}
    \intNum{\intProd{\Graph\phi}{\Diag X}} \coloneqq
    \tr{\cl{X\times_kX}{\intProd{\Graph\phi}{\Diag X}}}
    = \sum_{r\in\N} (-1)^r \trace{\phi^*}{H^r(X,\Ql)}
  \end{gather*}
\end{enumerate}
\end{proof}

\subsection{Other Cases}

\nocite{*}
\printbibliography
\end{document}
